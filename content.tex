\subsection*{2.3.3 (Ergänzung)}

Zusätzlich gibt es die Solo-Variante ``Nullensolo'' und die
Spielvariante ``Armut''. Das Nullensolo kann als Lust- oder Pflichtsolo
gespielt werden. Die Armut stellt eine neu geregelte Spielvariante dar.

\subsection*{2.3.4 (Ergänzung)}

Abweichend hiervon gelten die Karo Asse als sogenannte Schweinchen dann
und nur dann, wenn ein Spieler beide Karo Asse zum Zeitpunkt des ersten
Aufspiels auf seiner Hand vereint. In diesem Fall reihen sich die
Schweinchen in der Rangfolge der Trumpfkarten über die Herz 10 ein.

Sollten bei einer angenommenen Armut die beiden Karo Asse erst nach der
Spielfindung, also nach dem Tausch der Trumpfkarten zwischen dem
Spieler, der den Vorbehalt ange- meldet hat, und jenem welcher die Armut
annimmt, vereint sein, gelten sie dennoch als Schweinchen im weiteren
Spielverlauf.

Da es in nur manchen Spielen Schweinchen gibt, muss die Existenz der
Schweinchen vom Spieler, der die Schweinchen besitzt, angesagt werden.
Diese Ansage muss spätestens beim Einsatz (Legen) des ersten
Schweinchens gemacht werden. Wird die rechtzeitige Ansage der
Schweinchen versäumt, so gelten die Karo Asse als reguläre Füchse.

\subsection*{2.3.8 (neu)}

Bei der Spielvariante ``Nullensolo'' gilt die Rangfolge wie beim
Fleischlosen (Assesolo), siehe dazu Punkt 2.3.7.

\subsection*{2.3.9 (neu)}

Bei der Spielvariante ``Armut'' gilt die Rangfolge wie beim Normalspiel,
siehe dazu 2.3.4.

\subsection*{3.2.1 (Ersatz)}

Jener Mitspieler erhält Position 1, welcher zufällig die höchste
Spielkarte erhält. Dazu mischt ein beliebiger Mitspieler die
Spielkarten, lässt von einem beliebigen anderen Mitspieler vom
Kartendeck abheben (wie 3.3.3 und 3.3.5), und teilt allen Mitspielern im
Uhrzeigersinn direkt nacheinander, beginnend bei sich selbst, jeweils
eine Karte durch Abheben der obersten Karte vom Kartendeck zu.
Anschließend bilden alle Mitspieler einen Stich aus den ihnen
zugeteilten Karten. Dies können mehr als vier Karten sein. Dabei hat
der Kartengeber das Aufspiel. Es gilt die Kartenreihenfolge wie beim
Normalspiel, jedoch ohne Schweinchen. Der Gewinner des Stiches spielt an
Position 1.

\subsection*{4.1.5 (Ergänzung)}

Es gilt abweichend folgende Rangfolge beim Klären der Frage nach
Vorbehalten:

\begin{enumerate}
	\item Pflichtsolo
	\item Lustsolo
	\item Armut
	\item Hochzeit
\end{enumerate}

\section*{4.5 Armut (neu)}

\subsection*{4.5.1 (neu)}
 Ein Spieler ist genau dann berechtigt den Vorbehalt ``Armut''
anzumelden, wenn er außer einem möglichen Fuchs insgesamt höchstens
drei Trumpfkarten besitzt oder alle seine Trumpfkarten höchstens so
stark sind wie ein Fuchs. Dabei zählen Schweinchen (siehe 2.3.4) als
reguläre Trumpfkarten, sodass ein Spieler mit Schweinchen nur
höchstens eine andere Trumpfkarte auf seiner Hand halten darf, um Armut
anmelden zu können.

\subsection*{4.5.2 (neu)}

Stellt sich nach Klären der Vorbehaltsfrage heraus, dass die Variante
``Armut'' gespielt wird, sortiert der Spieler, der den Vorbehalt
angemeldet hat, alle Trumpfkarten und nur Trumpf- karten aus seiner Hand
auf einen separaten Stapel und legt diesen verdeckt in die Mitte des
Spieltisches.

\subsection*{4.5.3 (neu)}

Der linke Nachbar des Spielers, der Armut als Vorbehalt angemeldet hat,
muss entscheiden, ob er die Armut mitnimmt, ohne die Karten in der
Tischmitte anzusehen:

\begin{itemize}
	\item Die Armut wird mitgenommen: Der Spieler, welcher die Armut
		mitnimmt, bildet zusammen mit dem Spieler, welcher die Armut
		angemeldet hatte, die Re-Partei.
	\item Die Armut wird nicht mitgenommen: Der nächste linke Spieler
		muss entscheiden, ob er die Armut mitnimmt.
\end{itemize}

\subsection*{4.5.4 (neu)}

Ein Spieler, welcher die Armut mitnimmt, erhält die abgelegten
Trumpfkarten aus der Spiel- tischmitte auf seine Hand. Anschließend
wählt er genau so viele Karten aus seiner zu- sammengesetzten Hand, wie
er zusätzlich aufgenommen hat. Die Auswahl der Karten wird
ausdrücklich nicht eingeschränkt. Die ausgewählten Karten gibt der
Spieler anschließend verdeckt an seinen Partner und teil dabei allen
Spielern mit, wie viele Trumpfkarten er zurückgegeben hat.

\subsection*{4.5.5 (neu)}

Sollte kein Spieler die Armut annehmen, so wird das Spiel weder gespielt
noch gewertet und stattdessen werden die Karten vom gleichen Geber neu
ausgeteilt.

\subsection*{5.3.4 (Ergänzung)}

Eine gesondert abgelegte Karte zur Erinnerung an Sonderpunkte darf auch
offen abgelegt werden. Dabei ist darauf zu achten, dass die
Rekonstruktion der Stiche weiterhin gewähr- leistet werden kann.

\subsection*{6.3.4 (Ergänzung)}

Abweichend gilt beim Nullensolo: Nur die Kontra-Partei kann Absagen
machen. Dabei gelten folgende Absagezeitpunkte:

\begin{tabular}{ | l | l | l | l | }
  \hline
	``über 30'' abgesagt	& mit mindestens & 10	& Karten in der Hand \\ \hline
	``über 60'' abgesagt	& mit mindestens & 9	& Karten in der Hand \\ \hline
	``über 90'' abgesagt	& mit mindestens & 8	& Karten in der Hand \\ \hline
	``über 120'' abgesagt	& mit mindestens & 7	& Karten in der Hand \\ \hline
  \hline
\end{tabular}

\subsection*{7.2.1 (Ergänzung)}

Abweichend gilt beim Nullensolo: Die Re-Partei (der Solospieler) hat
gewonnen, wenn sie keinen Stich erhält, andernfalls hat die
Kontra-Partei gewonnen.


\subsection*{7.2.2 (Ergänzung)}

Abweichend gilt beim Nullensolo:

\begin{tabular}{ | l | l | l | }
  \hline
	(a)	& Gewonnen			& 1 Punkt als Grundwert \\ \hline
		& über 30 gespielt	& 1 Punkt zusätzlich \\ \hline
		& über 60 gespielt	& 1 Punkt zusätzlich \\ \hline
		& über 90 gespielt	& 1 Punkt zusätzlich \\ \hline
		& über 120 gespielt	& 1 Punkt zusätzlich \\ \hline
  \hline
\end{tabular}

Die Punkte in Tabelle (b) gelten unverändert. Jene in Tabelle (c)
finden keine Anwendung.


\begin{tabular}{ | l | l | l | }
  \hline
	(d)	& Es wurde von der Kontra-Partei: & \\ \hline
		& ``über 30'' abgesagt	& 1 Punkt zusätzlich \\ \hline
		& ``über 60'' abgesagt	& 1 Punkt zusätzlich \\ \hline
		& ``über 90'' abgesagt	& 1 Punkt zusätzlich \\ \hline
		& ``über 120'' abgesagt	& 1 Punkt zusätzlich \\ \hline
  \hline
\end{tabular}

Die Punkte in den Tabellen (e) und (f) finden keine Anwendung.

\subsection*{7.2.5 (neu)}

Es werden sogenannte Bockspiele berücksichtigt. In einem Bockspiel wird
der reguläre Spielwert verdoppelt. Wann ein Spiel als Bockspiel
gewertet wird, hängt von bestimmten Ereignissen ab.

\begin{itemize}
	\item Einer Reklamation beim Mischen, Abheben oder Austeilen wird
		stattgegeben. Das daran anschließende Spiel, also wenn der Geber
		wechselt, ist ein Bockspiel. Dies verhindert beim Spiel mit fünf
		Personen, dass der Verursacher von einem etwaigen doppelten
		Punkteverlust ausgeschlossen ist und somit kein Risiko dabei
		trägt, absichtlich die Karten falsch zu verteilen.
	\item Ein Spiel endet ohne zu vergebende Punkte. Das sofort neu
		ausgeteilte Spiel ist ein Bockspiel (siehe 7.3.5).
	\item Ein Vorbehalt ``Armut'' wird endgültig nicht mitgenommen. Das
		neu ausgeteilte Spiel ist ein Bockspiel.
\end{itemize}

Auch ein Spiel mit Vorbehalten kann ein Bockspiel sein.

Falls mehrere Ereignisse, egal ob verschieden oder wiederholt, das selbe
ausgeteilte Spiel zum Bockspiel machen, wird nur dieses eine Spiel als
Bockspiel gewertet (es werden keine Bockspiele nachgeholt) und es wird
nur doppelt gewertet (der Multiplikator wird nicht potenziert).

\subsection*{7.2.6 (neu)}

Wird ein Spieler vorgeführt, so werden die Punkte unter 7.2.2 (c) bis
(f) nicht gewertet.

\subsection*{7.3.5 (neu)}

Ein Spiel, das ohne Punkte endet, wird nicht gewertet. Stattdessen
werden die Karten neu ausgeteilt. Das neu ausgeteilte Spiel ist ein
Bockspiel (siehe 7.2.5).

\subsection*{10.1.2 (Ergänzung)}

Dies gilt in jedem Fall, also insbesondere auch bei Spielen mit
Vorbehalten.
