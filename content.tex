\section*{2.3 Rangfolge der Karten}

\subsection*{2.3.3 (Ergänzung)}

Zusätzlich gibt es die Solo-Variante \gdq{Nullensolo} und die
Spielvariante \gdq{Armut}. Das Nullensolo kann als Lust- oder Pflichtsolo
gespielt werden. Die Armut stellt eine neu geregelte Spielvariante dar.

\subsection*{2.3.4 (Ergänzung)}

Abweichend hiervon gelten die Karo Asse als sogenannte Schweinchen dann
und nur dann, wenn ein Spieler beide Karo Asse zum Zeitpunkt des ersten
Aufspiels auf seiner Hand vereint. In diesem Fall reihen sich die
Schweinchen in der Rangfolge der Trumpfkarten über die Herz 10 ein.

Sollten bei einer angenommenen Armut die beiden Karo Asse erst nach der
Spielfindung, also nach dem Tausch der Trumpfkarten zwischen dem
Spieler, der den Vorbehalt angemeldet hat, und jenem welcher die Armut
annimmt, vereint sein, gelten sie dennoch als Schweinchen im weiteren
Spielverlauf.

Da es in nur manchen Spielen Schweinchen gibt, muss die Existenz der
Schweinchen vom Spieler, der die Schweinchen besitzt, angesagt werden.
Diese Ansage muss spätestens beim Einsatz (Legen) des ersten
Schweinchens gemacht werden. Wird die rechtzeitige Ansage der
Schweinchen versäumt, so gelten die Karo Asse als reguläre Füchse.
% TODO bzgl. "Legen": hier sollte auf die neue Definition verwiesen werden
% die regelt, wann eine Karte als gelegt gilt.

\subsection*{2.3.8 (neu)}

Bei der Spielvariante \gdq{Nullensolo} gilt die Rangfolge wie beim
Fleischlosen (Assesolo), siehe dazu Punkt 2.3.7.

\subsection*{2.3.9 (neu)}

Bei der Spielvariante \gdq{Armut} gilt die Rangfolge wie beim Normalspiel,
siehe dazu 2.3.4.

\section*{3.2 Bestimmung der Plätze}

\subsection*{3.2.1 (Ersatz)}

Jener Mitspieler erhält Position 1, welcher zufällig die höchste
Spielkarte erhält. Dazu mischt ein beliebiger Mitspieler die
Spielkarten, lässt von einem beliebigen anderen Mitspieler vom
Kartendeck abheben (wie 3.3.3 und 3.3.5), und teilt allen Mitspielern im
Uhrzeigersinn direkt nacheinander, beginnend bei sich selbst, jeweils
eine Karte durch Abheben der obersten Karte vom Kartendeck zu.
Anschließend bilden alle Mitspieler einen Stich aus den ihnen
zugeteilten Karten. Dies können mehr als vier Karten sein. Dabei hat
der Kartengeber das Aufspiel. Es gilt die Kartenreihenfolge wie beim
Normalspiel, jedoch ohne Schweinchen. Der Gewinner des Stiches spielt an
Position 1.

\section*{3.3 Geben der Karten}

\section*{3.3.8 (Ergänzung)}

Eine Karte gilt beim Austeilen als aufgeworfen wenn sie benannt (siehe 6.1.3) ist.

\section*{3.3.11 (neu)}

Ein Fehler beim Geben der Karten gilt als unerheblicher Regelverstoß (siehe
9.2).

\section*{4.1 Die Vorbehaltsfrage}

\subsection*{4.1.5 (Ergänzung)}

Es gilt abweichend folgende Rangfolge beim Klären der Frage nach
Vorbehalten:

\begin{enumerate}
    \item Pflichtsolo
    \item Lustsolo
    \item Armut
    \item Hochzeit
\end{enumerate}

\section*{4.5 Armut (neu)}

\subsection*{4.5.1 (neu)}
 Ein Spieler ist genau dann berechtigt den Vorbehalt \gdq{Armut}
anzumelden, wenn er außer einem möglichen Fuchs insgesamt höchstens
drei Trumpfkarten besitzt oder alle seine Trumpfkarten höchstens so
stark sind wie ein Fuchs. Dabei zählen Schweinchen (siehe 2.3.4) als
reguläre Trumpfkarten, sodass ein Spieler mit Schweinchen nur
höchstens eine andere Trumpfkarte auf seiner Hand halten darf, um Armut
anmelden zu können.

\subsection*{4.5.2 (neu)}

Stellt sich nach Klären der Vorbehaltsfrage heraus, dass die Variante
\gdq{Armut} gespielt wird, sortiert der Spieler, der den Vorbehalt
angemeldet hat, alle Trumpfkarten und nur Trumpfkarten aus seiner Hand
auf einen separaten Stapel und legt diesen verdeckt in die Mitte des
Spieltisches.

\subsection*{4.5.3 (neu)}

Der linke Nachbar des Spielers, der Armut als Vorbehalt angemeldet hat,
muss entscheiden, ob er die Armut mitnimmt, ohne die Karten in der
Tischmitte anzusehen:

\begin{itemize}
    \item Die Armut wird mitgenommen: Der Spieler, welcher die Armut
        mitnimmt, bildet zusammen mit dem Spieler, welcher die Armut
        angemeldet hatte, die Re-Partei.
    \item Die Armut wird nicht mitgenommen: Der nächste linke Spieler
        muss entscheiden, ob er die Armut mitnimmt.
\end{itemize}

\subsection*{4.5.4 (neu)}

Ein Spieler, welcher die Armut mitnimmt, erhält die abgelegten
Trumpfkarten aus der Spieltischmitte auf seine Hand. Anschließend
wählt er genau so viele Karten aus seiner zusammengesetzten Hand, wie
er zusätzlich aufgenommen hat. Die Auswahl der Karten wird
ausdrücklich nicht eingeschränkt. Die ausgewählten Karten gibt der
Spieler anschließend verdeckt an seinen Partner und teil dabei allen
Spielern mit, wie viele Trumpfkarten er zurückgegeben hat.

\subsection*{4.5.5 (neu)}

Sollte kein Spieler die Armut annehmen, so wird das Spiel weder gespielt
noch gewertet und stattdessen werden die Karten vom gleichen Geber neu
ausgeteilt.

\section*{5.2 Bedienen}

\subsection*{5.2.2 (Ergänzung)}

Die Bedienpflicht gilt als verletzt sobald der Spieler, welcher gerade an der
Reihe war, eine Karte gelegt hat, die nicht der geforderten Art entspricht
während dieser Spieler noch mindestens eine Karte der geforderten Art auf der
Hand hält.

\subsection*{5.2.4 (Ergänzung)}

Wenn ein Spieler die Bedienpflicht verletzt hat und die erste Karte des nächstes
Stiches noch nicht gelegt ist, so muss die falsch bediente Karte und müssen alle
später gelegten Karten desselben Stiches zurückgenommen werden. Das Spiel wird
durch Legen einer regelkonformen Karte durch den Spieler, der die Bedienpflicht
verletzt hat, fortgesetzt.

Wurde die erste Karte des Stiches, der auf den Stich mit der falsch bedienten
Karte folgt, bereits gelegt, so finden keine Korrekturen statt. In diesem Fall
kann der Regelverstoß zwar reklamiert werden, das Spiel wird aber fortgesetzt.

Das Verletzten der Bedienpflicht kann nur dann reklamiert werden, wenn die
nächste Karte bereits regelkonform gelegt wurde oder der Stich, welcher die
falsch bediente Karte enthält, von einem beliebigen Spieler beim Gewinner des
Stiches abgelegt wurde.

\section*{5.3 Stiche}

\subsection*{5.3.4 (Ergänzung)}

Eine gesondert abgelegte Karte zur Erinnerung an Sonderpunkte darf auch
offen abgelegt werden. Dabei ist darauf zu achten, dass die
Rekonstruktion der Stiche weiterhin gewährleistet werden kann.

\section*{6.1 Definitionen}

\section*{6.1.1 (Ersatz)}

Eine Karte befindet sich \gdq{in der Hand} eines Spielers, wenn sie noch nicht
gespielt ist. Eine Karte \gdq{in der Hand} eines Spielers wird auch Handkarte
genannt.

\section*{6.1.3 (neu)}

Eine Karte gilt dann als benannt, wenn mindestens ein Spieler, der diese Karte
nicht besitzt, die Farbe (z.B. \gdq{Kreuz} -- \gdq{rot} oder \gdq{schwarz}
reicht nicht aus) oder den Kartenwert (z.B. \gdq{Dame} -- \gdq{Bild} reicht
nicht aus) nennen kann, weil er die Vorderseite dieser Karte gesehen hat.

\section*{6.1.4 (neu)}

Eine Karte gilt als gespielt bzw. als gelegt wenn sie vom Spieler, welcher an
der Reihe ist, ausgespielt wird und diese Karte als benannt gilt.

Eine Karte gilt als gefallen wenn sie von einem Spieler, welcher \textbf{nicht}
an der Reihe ist, sichtbar gemacht wurde und diese Karte als benannt gilt.
Sobald eine Karte gefallen ist muss sie allen Spielern sichtbar gemacht werden.
Anschließend wird die gefallene Karte wieder verdeckt in den Stapel der
Handkarten aufgenommen und gilt wieder als Handkarte.

Unter der Vorraussetzung dass kein Spieler glaubhaft einen Vorsatz unterstellt
kann das Fallenlassen\footnote{Das Verursachen, dass eine Karte als gefallen
gilt.} einer Karte nicht reklamiert werden.

\section*{6.3 Absagen und Absagezeitpunkt}

\subsection*{6.3.4 (Ergänzung)}

Abweichend gilt beim Nullensolo: Nur die Kontra-Partei kann Absagen
machen. Dabei gelten folgende Absagezeitpunkte:

\begin{tabular}{ | l | l | l | l | }
  \hline
    \gdq{über 30} abgesagt  & mit mindestens & 10   & Karten in der Hand \\ \hline
    \gdq{über 60} abgesagt  & mit mindestens & 9    & Karten in der Hand \\ \hline
    \gdq{über 90} abgesagt  & mit mindestens & 8    & Karten in der Hand \\ \hline
    \gdq{über 120} abgesagt & mit mindestens & 7    & Karten in der Hand \\ \hline
  \hline
\end{tabular}

\section*{7.2 Spielwerte}

\subsection*{7.2.1 (Ergänzung)}

Abweichend gilt beim Nullensolo: Die Re-Partei (der Solospieler) hat
gewonnen, wenn sie keinen Stich erhält, andernfalls hat die
Kontra-Partei gewonnen.

\subsection*{7.2.2 (Ergänzung)}

Abweichend gilt beim Nullensolo:

\begin{tabular}{ | l | l | l | }
  \hline
    (a) & Gewonnen          & 1 Punkt als Grundwert \\ \hline
        & über 30 gespielt  & 1 Punkt zusätzlich \\ \hline
        & über 60 gespielt  & 1 Punkt zusätzlich \\ \hline
        & über 90 gespielt  & 1 Punkt zusätzlich \\ \hline
        & über 120 gespielt & 1 Punkt zusätzlich \\ \hline
  \hline
\end{tabular}

Die Punkte in Tabelle (b) gelten unverändert. Jene in Tabelle (c)
finden keine Anwendung.

\begin{tabular}{ | l | l | l | }
  \hline
    (d) & Es wurde von der Kontra-Partei: & \\ \hline
        & \gdq{über 30} abgesagt    & 1 Punkt zusätzlich \\ \hline
        & \gdq{über 60} abgesagt    & 1 Punkt zusätzlich \\ \hline
        & \gdq{über 90} abgesagt    & 1 Punkt zusätzlich \\ \hline
        & \gdq{über 120} abgesagt   & 1 Punkt zusätzlich \\ \hline
  \hline
\end{tabular}

Die Punkte in den Tabellen (e) und (f) finden keine Anwendung.

\subsection*{7.2.5 (neu)}

Es werden sogenannte Bockspiele berücksichtigt. In einem Bockspiel wird
der reguläre Spielwert verdoppelt. Wann ein Spiel als Bockspiel
gewertet wird, hängt von bestimmten Ereignissen ab.

\begin{itemize}
    \item Einer Reklamation beim Mischen, Abheben oder Austeilen wird
        stattgegeben. Das daran anschließende Spiel, also wenn der Geber
        wechselt, ist ein Bockspiel. Dies verhindert beim Spiel mit fünf
        Personen, dass der Verursacher von einem etwaigen doppelten
        Punkteverlust ausgeschlossen ist und somit kein Risiko dabei
        trägt, absichtlich die Karten falsch zu verteilen.
    \item Ein Spiel endet ohne zu vergebende Punkte. Das sofort neu
        ausgeteilte Spiel ist ein Bockspiel (siehe 7.3.5).
    \item Ein Vorbehalt \gdq{Armut} wird endgültig nicht mitgenommen. Das
        neu ausgeteilte Spiel ist ein Bockspiel.
\end{itemize}

Auch ein Spiel mit Vorbehalten kann ein Bockspiel sein.

Falls mehrere Ereignisse, egal ob verschieden oder wiederholt, das selbe
ausgeteilte Spiel zum Bockspiel machen, wird nur dieses eine Spiel als
Bockspiel gewertet (es werden keine Bockspiele nachgeholt) und es wird
nur doppelt gewertet (der Multiplikator wird nicht potenziert).

\subsection*{7.2.6 (neu)}

Wird ein Spieler vorgeführt, so werden die Punkte unter 7.2.2 (c) bis
(f) nicht gewertet.

\section*{7.3 Spielliste}

\subsection*{7.3.5 (neu)}

Ein Spiel, das ohne Punkte endet, wird nicht gewertet. Stattdessen
werden die Karten neu ausgeteilt. Das neu ausgeteilte Spiel ist ein
Bockspiel (siehe 7.2.5).

\section*{7.4 Idiotensterne (neu)}

\subsection*{7.4.1 (neu)}

Bei der Bewertung werden sogenannte \gdq{Idiotensterne} berücksichtigt. Dabei
handelt es sich um einen ausgefüllten fünfeckigen Stern der auf dem
Wertungsblatt nahe dem Kürzel des Spielers notiert wird, wenn er an diesen
Spieler vergeben wird. Ein Spieler kann mehrere Idiotensterne auf sich vereinen.

\subsection*{7.4.2 (neu)}

Sollten mehrere Spieler am Ende einer Spielrunde (siehe 8.5.1) die gleiche
Punktezahl erreicht haben, so gilt die Anzahl der an die jeweiligen Spieler mit
gleicher Punktezahl vergebenen Idiotensterne als zweites Kriterium zur
Feststellung der Platzierung. Dabei unterliegt ein Spieler einem anderen Spieler
mit gleicher Punktezahl und weniger Idiotensternen.

\subsection*{7.4.3 (neu)}

Einen Idiotenstern erhält wer
\begin{itemize}
    \item{verursacht dass eine Karte als gefallen gilt.}
    \item{eine Spielaktion ausführt obwohl er nicht an der Reihe ist. Dies gilt
        insbesondere für das Beantworten der Vorbehaltsfrage wenn der
        Antwortende nicht an der Reihe ist oder das falsche Taufen eines
        Vorbehaltes.}
    \item{nicht regelkonform austeilt oder beim Austeilen eine Karte aufwirft.}
\end{itemize}

Für ein und dasselbe Fehlverhalten kann nur ein Idiotenstern vergeben werden.

\section*{10.1 Abweichungen an Fünfer-Tischen}

\subsection*{10.1.2 (Ergänzung)}

Dass der Geber keine Punkte erhält gilt in jedem Fall, also insbesondere auch
bei Spielen mit Vorbehalten und Bockspielen.

% TODO Eine Reklamation nicht zu ahnden oder eine Regel zu übergehen ist möglich
% dann und nur dann wenn alle Mitspieler (TODO Konkretisierung von "Mitspieler")
% sich dafür aussprechen. Es gilt ausdrücklich, dass ein einzelner Spieler die
% genaue Einhaltung der Regeln oder das Ahnden nach einer Reklamation verlangen
% kann.

% TODO Sobald ein Regelverstoß von mindestens einem Mitspieler angesprochen
% wird, muss jeder Mitspieler sofort entscheiden und bekannt geben ob der
% Verstoß reklamiert und geahndet werden soll. Eine nachträgliche Reklamation
% desselben Fehlverhaltens ist nicht möglich.

% TODO Absatz 9.3 über "unerhebliche Regelverstöße" gilt als nichtig. Eine
% Reklamation, wenn sie geahndet wird weil mindestens ein Mitspieler dies
% verlangt, wird immer als schwerwiegender Regelverstoß geahndet.

% TODO Strafpunkte im Falle von fünf Mitspielern (Spielvariante mit fünf
% Spielern) ist nicht geregelt...

% TODO Beim Vorführen werden keine An- oder Absagen gewertet.

% TODO Die Antwort auf eine Vorbehaltsfrage darf korrigiert werden, solange der
% nächste Spieler, der die Vorbehaltsfrage beantworten muss, diese Frage noch
% nicht beantwortet hat, bzw. solange, falls der Spieler mit dem Korrekturwunsch
% als letztes die Vorbehaltsfrage beantworten muss, bis der Spieler mit dem
% Aufspiel noch keine Karte gelegt hat. % TODO Armut

% Besser: Korrektur der Vorbehaltsfrage bis zur nächsten Aktion möglich. Nächste
% Aktion insbesondere Vorbehaltsfrage von nächstem Mitspieler beantwortet,
% Vorbehalt getauft oder Aufspieler hat Karte gelegt.

% TODO Pflichtsoli und Vorführespiele können keine Bockpspiele sein.

% TODO 9.7.1 findet keine Anwendung.
